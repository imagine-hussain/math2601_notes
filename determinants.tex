
\section{Determinants}

\paragraph{Formal Definition}
For a \( n \times n \) matrix \( A \): \[
    \det A = \sum_{\sigma \in \mathcal{S}_n}
    \sign(\sigma) a_{1, \sigma(1)} a_{2, \sigma(2)} \cdots a_{n, \sigma(n)}
\]

\paragraph{Defining \(\mathcal{S}_n\) and Sign}
In the above definition \(\mathcal{S}_n\) is the group of permutations of \( n \)
items. A permuation is even or odd if the number of inversions in that
permuation are even or odd.
Then, \(\sign(\sigma) = 1\) if \(\sigma\) is even and \(\sign(\sigma) = -1\) if
\(\sigma\) is odd.


\paragraph{Composability of Permutations}
Suppose that \( \alpha, \beta \in \mathcal{S}_n \) then, \[
    \sign(\alpha \circ \beta) = \sign(\alpha) \sign(\beta)
    \quad\text{and}\quad
    \sign(\alpha^{-1}) = \sign(\alpha)
\]

\paragraph{Transposition}
A transposition on \( \sigma \in \mathcal{S}_n \) is a permutation that swaps
two elements. All transpositions are odd permutations and thus reverse the sign.

\paragraph{Properties on the Determinant}
For \( A \in \mathbb{p, p} \)
\begin{enumerate}
    \item \(\det A = \det A^T\) and \( \det A^* = \overline{\det A} \).
    \item If \textit{any} row or column is zero, then \( \det A = 0 \).
    \item Applying a permuation to the rows is equivalent to multiplying the
        determinant by the sign of the permutation.
    \item \( \det A = 0 \) if any only if \( A \) is invertible.
    \item Adding a multiple of a row or column to different row or column
        does not change the determinant.
    \item As a map on the rows / columns of \( A \), the determinant is multilinear
    and alternating. Thus, multiple any row or column by a scalar \( \lambda \) multiplies
    the determinant by \( \lambda \). Thus, \( \det (\lambda A) = \lambda^p \det A \).
\end{enumerate}

\paragraph{Minor of a Matrix}
For a \( p\times p \) matrix \( A \), the \((i,j)\)-minor,
denoted \( A_{ij} \) is the resulting matrix obtained by deleting the \(i\)th row and \(j\)th column from \( A \).
Suppoes that the \( i \)-th row or \( j \)-th column is
all zeroes except for \( a_{ij} \), then \[
    \det A = {(-1)}^{i + j} a_{ij} \det(A_{ij})
.\]
The right hand-side is called the \textit{cofactor}
of element \( a_{ij} \).

\paragraph{Cofactor and Determinant}
Fix any \( i \). Then, the determinant is the sum of all
cofactors \( c_{ij} \).
 
\paragraph{Finding Determinants Efficiently}
\begin{enumerate}
    \item Add multiples of rows to each other until there
    is a row or column with only one non-zero entry.
    \item Use the result for a minor of a matrix to
        calculate the determinant of a \( (p-1)\times(p-1) \)
        matrix.
    \item Repeat recursively till a \( 2\times 2 \) matrix.
\end{enumerate}
If however, a matrix can be reduced to an upper triangular form,
that is more efficient as the determinant is the product
of the diagonal entries.

\paragraph{Elementary Matrices}
Elementary matrices are matrices that perform one of the following actions
\begin{enumerate}
    \item Swap two rows: \( \det E = -1 \).
    \item Multiplies a row by a scalar \( \lambda \): \( \det E = \lambda \).
    \item Add a multiple of one row to another: \( \det E = 1 \).
\end{enumerate}

For any invertible \( A \), it may be seen as a
composition of elementary matrices and the determinant,
their product.

\paragraph{Determinant and Invertibility}
Note that \( \det(AB) = \det(A) \det(B) \).
Applying the previous result of viewing matrices as
compositions of elementary functions, the determinant is
a similarity invariant.
An inverse exists if and only if, the determinant is non-zero.
Further, if \( A \) is invertible, then \[
    \det (A^{-1}) = \frac{1}{\det A}
\]

\paragraph{Adjugate / Classical Adjoint / Adjunct}
For a matrix \( A \), the ajunct is the transpose of the
matrix of cofactors. That is, \( \mathrm{adj}(A)_{ij} = c_{ji} \).
This is useful (though tediuos) since \[
    A^{-1} = \frac{\textrm{adj} A}{\det A}
.\]

