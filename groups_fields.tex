
\section{Groups and Fields}

\subsection{Groups}

\paragraph{Definition of Groups}
A group \(G\) is a non-empty set with a binary operation defined on it.
It must satisfy the following four properties:
\begin{enumerate}
    \item \textbf{Closure:} For all \(a, b \in G\), a composition \(a*b\) is
    defined and in \(G\).
    \item \textbf{Associativity:} (a*b)*c = a*(b*c).
    \item \textbf{Identify:} There exists an \(e\in G\) such that
    \(a*e = e*a = a\)  for all \(a\in G\).
    \item \textbf{Inverse:} For all \(a \in G\), there exists an \(a'\)
    such that \(a*a' = a'*a = e\).
\end{enumerate}

\paragraph{Groups Order and Pair}
Groups are actually pairs of objects. The first is the set of elements in
the group and the second, the operation defined on the group.
Therefore, groups may be written as \((G, *)\).

If \(G\) is finite, then the order of \(G\), that is \(|G|\), is the
number of element in \(G\).

\paragraph{Abelian Groups}
A group is abelian if the operation is \textit{commutative}.
That is,
\[a*b = b*a \quad \forall a,b\in G.\]

\paragraph{Notes on the Composition}

Observe that the composition is actually a function \(*: G\times G \to G\).
\(a*b\) is simply a more convenient notation than \(*(a, b)\).

Though the operation \(*\) is not restricted, it is often one of addition
(only for abelian groups),
multiplication (\(\times\), often written as juxtaposition) or,
composition of functions.

\paragraph{Notation for Repeated Composition}
We may often use power notation for repeated applications of a composition.
That is, \(a*a*\dots*a\) (with \(n\) compositions) may be written as \(a^n.\)

Suppose that instead we are using \(+\) as the group operation, then
\(a+a+\dots+a\) (added \(n\) times), may be written as
\(na\). Do note that this is \underline{not} multiplication.

\paragraph{Trivial Groups}
The trivial group consists of exactly one element, the identity.
That is, \(\{e\}\).
Since the empty set cannot be a group, as there is required to be at least
one element in a group, the trivial group is the smallest group that exists.


\paragraph{Examples of Groups}
\((\mathbb{Z}, +)\) is an abelian group under the usual addition operation.
However, \((\mathbb{Z}, \times)\) is not a group, since the inverse property
cannot be satisfied.
Similarly, \((\mathbb, \times)\) is also not a group as \(0\) has no multiplicative
inverse. However \((\mathbb{R}\backslash 0)\) is a group.

For an integers in the set \(\mathbb{Z}_m = \{0, 1, 2,\dots, m-1\}\)
is a group under addition, modulo \(m\).

\paragraph{Function Composition and Groups}
For any \(S\), the set \(F\) of bijective functions
\(f: S\to S\) is a group under composition but,
it is not necessarily abelian.

    \subparagraph{Proof}
    Composing two bijections gives a bijection so, the operation is closed.

    Associativity is of composition follows as
    \[
        (f\circ (g\circ h))(x) = f(g(h(x))) = (f\circ g)\circ h(x).
    \]
    The identity function is \(e(x) = x\) is a clear bijection.
    The inverse exists by definition of the bijection.

\paragraph{More Properties of Groups}
\begin{itemize}
    \item There is only \textit{one} inverse of each element. That is,
    the inverse is unique.
    \item For all \(a\in G\), \(\left(a^{-1}\right)^{-1} = a\)
    \item For all \(a, b \in G\), \(
        \left(a*b\right)^{-1} = b^{-1} * a^{-1}.
    \)
    \item let \(a, b, c \in G\). Then if \(
        a * b = a * c
    \), then \(b = c\).
    We may think of this as the cancellation property.
\end{itemize}

\paragraph{Permutation Groups}
Let \(\Omega_n = \{1, 2, \dots, n\}\). As a ordered set, \(\Omega_n\)
has \(n!\) permutations.

We may think of these permutations as being functions
\(f: \Omega_n \to \Omega_n\). Clearly, these are bijections.

Observe that the set \(S_n\) of all permutations forms a group under the
composition of order \(n\) as, the set of all bijections on a set is a
group.

We may write these permutations \(f\) as a matrix where, the \(i,j\)
entry represents how what element is mapped to the \(j\)-th index, by \(f_i\).

\paragraph{Small Finite Groups}
We may visualise these with a multiplication table where,
the row element is multiplied on the left of the column element.

In a multiplication table of a finite group, each row must be a permutation
of the elements of the group. Otherwise, if there was a repetition in a row
then \(xa = xb\) implies \(a = b\) by the cancellation property.
Thus, each element occurs no more than once is a row.

If \(a^2 = a\) then, by cancellation property, \(a = e\). So,
the identity must be the only element that is fixed.


\subsection{Fields}

\paragraph{Definition of Fields}
A field is a set \(\mathbb{F}\) with two binary operations on it,
addition \((+)\) and, multiplication, \((\times)\) such that, the following
hold
\begin{enumerate}
    \item \((\mathbb{F}, +)\) is an abelian group.
    \item \((\mathbb{F}^*\backslash \{0\}, \times)\) is an abelian group, where
    \(0\) is the additive identity.
    \item The distributive laws \(a\times (b+c) = (a\times b) + (a\times c)\)
    and \((a+b)\times c = a\times c + b\times c\) hold.
\end{enumerate}


\paragraph{Fields and Notation}
\begin{itemize}
    \item Under the obvious operations, typically refer to the field as
    just \(\mathbb{F}\).
    \item We use juxtaposition for multiplication under fields and, \(1\) as
    the multiplicative identity and often \(0\) as the additive identity.
    \item By our definition of fields as groups, it is equivalent to say
    that if \(\mathbb{F}\) is a field then, it satisfies the
    \(12 = 5 + 5 . + 2\) number laws.
    \item The smallest possible fields only has two elements, the
    multiplicative and additive identity. That is, \(\{0, 1\}\).
    \item We let \(-b\) be the inverse of \(b\) under addition and may write
    \(a + (-b)\) as \(a - b\) as a shorthand. Similarly, we may write
    \(\frac{a}{b}\) rather than \(ab^{-1}\) where \(b^{-1}\) is the
    multiplication inverse and \(b\neq 0\).
\end{itemize}

\paragraph{Finite Fields}
The only finite fields that exists are those of the size \(p^k\) for some
positive integer \(k\)  and prime \(p\) (also known as, the characteristic
of the field).

These may be called \textit{Galois fields} of size \(p^k\). That is,
\(GF(p^k)\).
Note that \(GF(p^k) \neq \mathbb{Z}_{p^k}\) unless \(k = 1\).

\paragraph{Properties of Fields}
If \(\mathbb{F}\) is a field and \(a, b, c \in \mathbb{F}\) then,
\begin{itemize}
    \item \(a0 = 0\)
    \item \(a(-b) = -(ab)\)
    \item \(a(b-c) = ab - ac\)
    \item If \(ab = 0\) then either \(a = 0\) or, \(b = 0\).
\end{itemize}

\subsection{Subgroups and Subfields}

\paragraph{Defining Subgroups}
Let \((G, *)\)  be a group and \(H\) be a non-empty subset of \(G\).
Suppose that \((H, *)\) satisfies the requirements of a group, then it is a
subgroup of \(G\).

We may write \(H \leq G\) such that \(H\) inherits the group structure from
\(G\).

\paragraph{The Subgroup Lemma}
Let \((G, *)\) be a group and \(H\) a non-empty subset of \(G\).
\(H\) is a subgroup if and only if
\begin{enumerate}
    \item \(a*b \in H \quad \forall a,b \in H\)
    \item \(a^{-1} \in H \quad \forall a\in H\)
\end{enumerate}
That is, \(H\) is closed under \(*\) and \( ^{-1}\).

Associativity for the subset follows from associativity of the group
structure. The identity also follows from the closure under an inverse
and multiplication since \(a^{-1}*a = e \in H\).

Note that every non-trivial group \(G\) will have two subgroups.
They are the \(\{e\}\) and \(G\).

\paragraph{General Linear Groups}
Let \(n\) be an integer such that \(n \geq 1\). The set of invertible
\(n\times n\) matrices over \(\mathbb{F}\) is a group under the operation
of matrix multiplication.
This is a special case of a bijection function \(f: S\to S\) with
\(S = \mathbb{F}^n\). This group will be non-abelian if \(n > 1\).
This group is names the \textit{General Linear Group}, denoted as
\(GL(n, \mathbb{F})\)

\paragraph{Special Linear Group}
The special linear groups are a subset of the general linear groups
denoted as \(SL(n, \mathbb{F})\) with the requirement that the matrices
all have a determinant of \(\mathbb{R}\).

\paragraph{Orthogonal Matrix Group}
The set of \(n\times n\) orthogonal matrices over \(\mathbb{F}\) is a subgroup of
\(GL(n, \mathbb{F})\).

That is, \(O(n) \leq GL(n, \mathbb{R})\).

There also exists \(SO(n) = O(n) \cap SL(n, \mathbb{R})\) which is the
intersection of the orthogonal matrices and special matrices which is also
a group.

\paragraph{Subfields}
Let \((\mathbb{F}, +, \times)\) be a field and
\(\mathbb{E}\subseteq \mathbb{F}\) such that \(\mathbb{E}\) is also a
field under the same operations.

Then, \((\mathbb{E}, +, \times)\) is a subfield of \(\mathbb{F}\).
Equivalently, \(\mathbb{E}\leq \mathbb{F}\).

\paragraph{Subfield Lemma}
Let \(\mathbb{E} \neq \{0\} \) be a non-empty subset of a field
\(\mathbb{F}\).
Then, \(\mathbb{E}\) is a subfield of \(\mathbb{F}\) if and only if,
for all \(a, b \in \mathbb{E}\),

\begin{enumerate}
    \item \(a + b \in \mathbb{E}\),
    \item \(- b \in \mathbb{E}\),
    \item \(a\times b \in \mathbb{E}\),
    \item \(b^{-1} \in \mathbb{E}\) given \(b \neq 0\).
\end{enumerate}

The distributive laws are inherited from \(\mathbb{F}\) and need no checking.
The rest of the proof may follow from applications of the subgroup lemma
to each operation \(\mathbb{E}, +\) and \(\mathbb{E}, \times\).

\paragraph{Cool Rational + Irrational Alpha Field}
Let \(\alpha\) be any non-rational real or complex number.
We may define \( \mathbb{Q}(\alpha)\) to be the smallest field containing both
\(\mathbb{Q}\) and \(\alpha\).

The smallest such field is of the form \(\{
    a + b\alpha : a, b \in \mathbb{Q}
\}\).

\subsection{Morphisms}
Morphisms are the \textit{nice} maps between the members.

\paragraph{Homomorphism Definition}
Let \((G, *)\) and \((H, \circ)\) be two groups.
A (group) homomorphism from \(G\) to \(H\) is a map
\(\phi : G \to H\) that respects the two operations.

That is, \[
    \phi(a*b) = \phi(a) \circ \phi(b) \quad \forall a,b \in G
\]

\paragraph{Isomorphism}
An isomorphism is a bijective homomorphism \(\phi : G\to H\).
The groups are then isomorphic. That is, \(G \cong H\).

In terms of group theory, if two groups are isomorphic then,
they are effectively the same group. Isomorphism is an equivalence
relation on groups.

\paragraph{Isomorphism Lemmas}
let \((G, *)\) and \((H, \circ)\) be two groups and \(\phi\) a homomorphism
between them.
Then,
\begin{itemize}
    \item \(\phi\) maps the identity of \(G\) to the \( G \) to the
    identity of \(H\).
    \item \(\phi\) maps the inverses to inverses. That is,
    \(\phi(a^{-1}) = (\phi(a))^{-1}\), for all \(a \in G\).
    \item if \(\phi\) is a isomorphism from \(G \to H\) then,
    \(\phi^{-1}\) is an isomorphism from \(H \to G\).
\end{itemize}

\paragraph{Images and Kernel Definition}
Let \(\phi: G\to H\) be a group homomorphism with, \(e'\) the identity
of \(H\).

The kernel of \(\phi\) is the set
\[
    \ker(\phi) = \{
        g\in G : \phi(g) = e'
    \}.
\]
Observer that \(\ker \phi \leq G\).

The image of \(\phi\) is the set \[
    \mathrm{im} (\phi) = \{
        h\in H : h = \phi(g), \text{ some } g\in G
    \}.
\]

Note that \(\mathrm{im}\phi \leq G\).

\paragraph{One-to-One Homomorphisms}
A homomorphism is one-to-one if and only if \(\ker \theta = \{e\}\),
where \(e\) is the identity of \(G\).

If \(\phi\) is one-to-one then, \(\textrm{im}(\phi)\) is isomorphic to
\(G\).

\paragraph{Group Homomorphisms and General Linear Group}
It is common to seek a homomorphism
\(\phi: G \to GL(n, \mathbb{F})\) for some \(n\) and field \(\mathbb{F}\).

If \(\phi\) is one to one (each element maps to a unique matrix), then
the representation is faithful.

\subparagraph{Example of Group Homomorphisms on \(GL\)}
Consider \(C_4 = \{e, a, a^2, a^3\}\) (\(a^4 = e\)).
Let \(J = \begin{pmatrix}
        0  & 1 \\
        -1 & 0
    \end{pmatrix}.
\)
Define \(\phi: C_4 GL(2, \mathbb{R})\) by
\(\phi(a^k) = J^k\).

Then, \(\phi\) is a faithful representation of \(C_4\) on \(\mathbb{R}^2\).

Often, we may represent to the image subgroup of matrices as the group
itself.

\paragraph{Cayley's Theorem}
We may apply the lemma that \(\textrm{im}(\phi)\) is a subgroup, to the
case where \(G\) is finite and \(H\) is \(S_n\) for some \(n\).

In this case, we get a permutation representation of the group \(G\)
as a subgroup of \(S_n\).

Cayley's theorem states that every finite group \(G\) has a representation
as a subgroup of \(S_n\) for \(n \leq \abs{G}\).
That is, \(S_n\) and its subgroups are all that exists in the case of
finite groups.
