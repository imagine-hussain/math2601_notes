
\section{Linear Transformations}

\subsection{Linear Transformations}

\paragraph{Linear Transformations}
The morphisms (nice maps) respect both operations on a vector space.
Suppose that \(V, W\) are vector spaces over a field \(\mathbb{F}\).
Then, a function \(T: V\to W\) is a linear transformation only if
\[
    T(\vv + \vu)  = T(\vv) + T(\vu),
    \text{ and }
    T(\lambda \vv) = \lambda T(\vv),
\] for all \(\vv, \vu \in V\) and \(\lambda \in \mathbb{R}\).

\paragraph{Linearity Test Lemma}
A function \(T: V \to W\) between vector spaces is linear if and only if
\[
    T(\lambda \vu + \vv) = \lambda T(\vu) + T(\vv).
\]

\paragraph{Linear Transformations are Vector Spaces}
Let \(V, W\) be vector spaces over \(\mathbb{F}\).
Then, the set \(L(V, W)\) of linear transformations from
\(V\) to \(W\) is a vector space under the operations
\[
    (S + T)(\vv) = S(\vv) + T(\vv),
    \text{ and }
    (\lambda S)(\vv) = \lambda S(\vv).
\]

\paragraph{Composition Of Linear Transformations are Vector Spaces}
Let \(T: V\to W\) and \(S: W\to X\) be linear maps between vector spaces.
Then the following composition is also linear \(S\circ T : V\to X\).

\paragraph{Linearity of Inverse}
Let \(T: V \to W\) be an invertible linear map between vector spaces over
\(\mathbb{F}\).
Then, \(T^{-1}: V \to W\) is linear.

\paragraph{Invertible Linear are Groups}
The invertible linear maps \(L(V, V)\) form a group under composition.
Note that composition of maps is always associative so and the
inverse exists by definition of \(L(V, V)\), only
closure and the identity need to be proved.

Closure exists since composition of linear transformations are vector spaces.
The identity map is linear and clearly invertible and so, also exists in
the group.

\paragraph{Taking Coordinates is Linear}
Let \(V\) be a finite-dimensional vector space over \(\mathbb{F}\) with a basis
\(\mathcal{B} = \{\vv_1, \dots \vv_n\}\).
Define \(S: V \to \mathbb{F}^n\) as \( S(\vx) = [\vx]_b \); that is, \(S\) reprsents
a change of coordinated to the basis \(\mathcal{B}\).
Then, \(S\) is linear.

%
%
%
\subsection{Kernel and Image}

\paragraph{Definiton: Kernel and Image}
Let \(T: V\to W\) be a linear transformation.
The \textit{kernel} is the set such that
\[
    \ker T = \{\vv \in V : T(\vv) = \zero\}.
\]

If \(U \leq V\) then, the \textit{image} of \(U\) is the set \[
    T(U) = \{T(\vu) : \vu \in U \}.
\]

Also, the \textit{image} of \(T\) (or range) is defined as the image of all
\(V\) so \(\img(T) = T(V)\).

\paragraph{Kernel and Image of Linear Transformations}
Let \(T: V \to W\) be a linear transformation between vector spaces over
\(\mathbb{F}\) where \(U \leq V\).
Then,
\begin{enumerate}
    \item \(\ker T\) is a subspace of \(V\)
    \item \(T(U)\) is a subspace of \(W\) and thus, \(\img(T) \leq W\).
    \item If \(U\) is finite-dimensional, so is \(T(U)\) and thus if \(V\)
    is finite dimensional, so is \(\img(T)\).
\end{enumerate}

\paragraph{Rank and Nullity}
Let \(T\) be a linear transformation. The nullity is the dimension
of the kernel of \(T\).
The rank is the dimension of the image of \(T\).

\paragraph{Rank-Nullity Theorem}
If \(V\) is a finite dimensional vector space over \(\mathbb{F}\) and
\(T: V \to W\) is linear then,
\[\rank(T) + \nullity(T) = \dim(V).\]

