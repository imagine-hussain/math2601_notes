
\section{Linear Transformations}

\subsection{Linear Transformations}

\paragraph{Linear Transformations}
The morphisms (nice maps) respect both operations on a vector space.
Suppose that \(V, W\) are vector spaces over a field \(\mathbb{F}\).
Then, a function \(T: V\to W\) is a linear transformation only if
\[
    T(\vv + \vu)  = T(\vv) + T(\vu),
    \text{ and }
    T(\lambda \vv) = \lambda T(\vv),
\] for all \(\vv, \vu \in V\) and \(\lambda \in \mathbb{R}\).

\paragraph{Linearity Test Lemma}
A function \(T: V \to W\) between vector spaces is linear if and only if
\[
    T(\lambda \vu + \vv) = \lambda T(\vu) + T(\vv).
\]

\paragraph{Linear Transformations are Vector Spaces}
Let \(V, W\) be vector spaces over \(\mathbb{F}\).
Then, the set \(L(V, W)\) of linear transformations from
\(V\) to \(W\) is a vector space under the operations
\[
    (S + T)(\vv) = S(\vv) + T(\vv),
    \text{ and }
    (\lambda S)(\vv) = \lambda S(\vv).
\]

\paragraph{Composition Of Linear Transformations are Vector Spaces}
Let \(T: V\to W\) and \(S: W\to X\) be linear maps between vector spaces.
Then the following composition is also linear \(S\circ T : V\to X\).

\paragraph{Linearity of Inverse}
Let \(T: V \to W\) be an invertible linear map between vector spaces over
\(\mathbb{F}\).
Then, \(T^{-1}: V \to W\) is linear.

\paragraph{Invertible Linear are Groups}
The invertible linear maps \(L(V, V)\) form a group under composition.
Note that composition of maps is always associative so and the
inverse exists by definition of \(L(V, V)\), only
closure and the identity need to be proved.

Closure exists since composition of linear transformations are vector spaces.
The identity map is linear and clearly invertible and so, also exists in
the group.

\paragraph{Taking Coordinates is Linear}
Let \(V\) be a finite-dimensional vector space over \(\mathbb{F}\) with a basis
\(\mathcal{B} = \{\vv_1, \dots \vv_n\}\).
Define \(S: V \to \mathbb{F}^n\) as \( S(\vx) = [\vx]_b \); that is, \(S\) reprsents
a change of coordinated to the basis \(\mathcal{B}\).
Then, \(S\) is linear.

%
%
%
\subsection{Kernel and Image}

\paragraph{Definiton: Kernel and Image}
Let \(T: V\to W\) be a linear transformation.
The \textit{kernel} is the set such that
\[
    \ker T = \{\vv \in V : T(\vv) = \zero\}.
\]

If \(U \leq V\) then, the \textit{image} of \(U\) is the set \[
    T(U) = \{T(\vu) : \vu \in U \}.
\]

Also, the \textit{image} of \(T\) (or range) is defined as the image of all
\(V\) so \(\img(T) = T(V)\).

\paragraph{Kernel and Image of Linear Transformations}
Let \(T: V \to W\) be a linear transformation between vector spaces over
\(\mathbb{F}\) where \(U \leq V\).
Then,
\begin{enumerate}
    \item \(\ker T\) is a subspace of \(V\)
    \item \(T(U)\) is a subspace of \(W\) and thus, \(\img(T) \leq W\).
    \item If \(U\) is finite-dimensional, so is \(T(U)\) and thus if \(V\)
    is finite dimensional, so is \(\img(T)\).
\end{enumerate}

\paragraph{Rank and Nullity}
Let \(T\) be a linear transformation. The nullity is the dimension
of the kernel of \(T\).
The rank is the dimension of the image of \(T\).

\paragraph{Rank-Nullity Theorem}
If \(V\) is a finite dimensional vector space over \(\mathbb{F}\) and
\(T: V \to W\) is linear then,
\[\rank(T) + \nullity(T) = \dim(V).\]

\subparagraph{Rank and Nullity - One to One}
A linear map \(T\) is one-to-one if and only if
\(\nullity(T) = 0\).

\paragraph{Bijetions, Injections and Surjections}
Let \(V, W\) be vector spaces over \(\mathbb{F}\) and suppose
\(\dim(V) = \dim(W)\) which are both finite.
Let \(T: V \to W\) be linear.
The following are equivalent
\begin{enumerate}
    \item \textbf{Bijective:} \(T\) is invertible,
    \item \textbf{Injective:} \(T\) is one-to-one,
    \item \textbf{Surjective:} \(T\) is onto. That is, \(\rank(T) = \dim(V)\).
\end{enumerate}

\paragraph{Isomorphic Maps}
An invertible linear map \(T: V \to W\) is an isomorphism of
vector spaces \(V\) and \(W\).

Isomorphism is an equivalence relation on vector spaces.
\begin{enumerate}
    \item \textbf{Reflexive:} The identity map \(\mathrm{id}: V\to V\) is
        an isomorphism from \(V\) to itself.
    \item \textbf{Symmetric:} If \(T: V \to W\) is an isomorphism then,\
        \(T^{-1}\) is an isomorphism from \(W \to V\).
    \item \textbf{Transitive:} If \(T: V \to W\) is an isomorphism and
        \(S: W\to X\) is an isomorphism then
        \(S \circ T : V \to X\) is an isomorphism.
\end{enumerate}

\paragraph{Easy Test for Isomorphism}
Finite dimension spaces \(V, W\) are isomorphism over 
\(\mathbb{F}\) if and only if,
\[ \dim(V) = \dim(W).\]

\paragraph{Isomorphism: Coordinates and Matrices}
Taking coordinates in some chosen basis is an isomorphism.
All isomorphisms to \(\mathbb{F}^p\) can be described in this way.

\subparagraph{Isomorphism of Row and Column Matrices}
Note that all the vector spaces
\(\mathbb{F}^p, M_{p, 1}(\mathbb{F}), M_{1, p}(\mathbb{F})\)
are all isomorphic.

\subsection{Spaces Associated with Matrices}

\paragraph{Kernel, Image Rank - Matrix as Map}
Let \(A\) be a \(p\times q\) matrix over a field \(\mathbb{F}\).
Define a map \(T: \mathbb{F}^q \to \mathbb{F}^p\)
as \(T(\vx) = A\vx\).

Then, the kernel, rank, image and nullity of \(A\) are by definition, the same
as those defined under the map \(T\).

\paragraph{Image}
Suppose \(A\) has columns \(c_1, \dots, c_q \in \mathbb{F}\).
Then,
\begin{align*}
    \img(A) &= \{A \vx : x \in \mathbb{F}^q\} \\
            &= \{x_1, \vc_1 + \dots, x_q \vc_q : x_i \in \mathbb{F}\}
            &= \spans(\{\vc_1, \dots, \vc_nq\}).
\end{align*}
That is, \(\img(A)\) is the space spanned by the columsn of \(A\), also
knows as the column space of \(A\).
This may be denoted as \(\col(A) \leq \mathbb{F}\).

The rank of \(A\) is the dimension of the column space of \(A\).

\paragraph{Rank-Nullity Theorem: Matrices}
For \(A\in M_{p, q} \), let \(q\) be the number of columns of \(A\).
Then, \[
    \rank(A) + \nullity(A) = q
.\]

\paragraph{Row Space}
The row space of \(A\), \(\row(A)\) is the space spanned by the rows of \(A\).
It follows that \[
    \row(A) = \col(A^T) = \img(A^T).
.\] 

\paragraph{Rank of Matrix}
The rank of the matrix is identical to the column space of the matrix.
The row-rank (dimension of the row space) is equivalent to the rank.

\subsection{Matrix of a Linear Map}

\paragraph{Matrices of Linear Maps}
Let \(V, W\) be two finite dimensional vector spaces over \( \mathbb{F} \).
Suppose that \(\dim(V) = q\) and \(V\) and basis \(\mathcal{B}\).
Also suppose that \(W\) has basis \(\mathbb{C}\) and \(\dim(W) = p\).

If \(T: V \to W\) is a linear map then, there exists a unique
\(A \in M_{p, q} \mathbb{F}\) with
\[
    [T(\vv)]_\mathcal{C} = A[\vv]_\mathcal{B}
.\]

Conversely, for any \(A\in M_{p, q}(\mathbb{F})\), the equation
above defines a unique linear map from \(V \to W\).

\subparagraph{Notation}
We call the matrix \(A\) above, the matrix of \(T\) with respect
to \(\mathcal{B}\) and \(\mathcal{C}\)
A useful notation for this the matrix \([T]_\mathcal{C}^\mathcal{B}\).
Then, the equation from the theorem takes the following form: \[
    [T(\vv)]_\mathcal{C}
    =
    [T]_\mathcal{C}^\mathcal{B} [\vv]_\mathcal{B}
.\]
You may think of the \(\mathcal{B}\) as cancelling out. This is not very rigourous.
% \paragraph{Finding \([T])_\mathcal{C}^\mathcal{B}\)}

\paragraph{Composition of Linear Maps as Matrices}
Let \(T: V \to W\) and \(S: W \to X\) be linear maps betweeen vector spaces.
Suppose \(V, W, X\) have bases \(\mathcal{A}, \mathcal{B}, \mathcal{C}\)
respectively.
Then, the matrix \(S \circ T : V \to X\) is the product of teh matrices of
\(T\) and \(S\) with the appropriate bases as follows:
\[
   [S \circ T]_\mathcal{C}^\mathcal{A}
   =
   [S]_\mathcal{C}^\mathcal{B} \cdot [T]_\mathcal{B}^\mathcal{A}
.\]

\paragraph{Inverting Matrices as Transformations}
If \(T: V \to W\) is linear and invertible then, the matrix of \(T^{-1}\)
is the inverse of the matrix of \(T\).
Alternatively \[
    [T^{-1}]_\mathcal{B}^\mathcal{C}
    =
    \left( [T]_\mathcal{B}^\mathcal{C} \right)^{-1}
.\]

\paragraph{Change of Basis Matrix}
Suppose that a vector space has two bases \(\mathcal{B}\) and \(\mathcal{C}\).
Then the matrix \([id]_\mathcal{C}^\mathcal{B}\) is called the
change of basis matrix (from \(\mathbb{B} \to \mathbb{C}\)).

This may be used to change coordinates as follows: \[
    [\vv]_\mathcal{C}
    =
    [\id]_\mathcal{C}^\mathcal{B} [\vv]_\mathcal{B}
.\]
It may often be easier to find the basis change in the other direction and
then take the inverse.

\paragraph{Finding a Transformation Matrix}
To find the matrix \(T: V \to W\) with respect to bases \(\mathcal{B}\)
as the domain and \(\mathcal{C}\) in the codomain, \(T\) is found through the
following three mappings.

For any  \(\vv \in V\):
\begin{enumerate}
    \item Calculate \(\id_V (\vv) = \vv\) where \(\id_V\) is the identity
        map on \(V\).
    \item Calculate \(T(\id_V(\vv)) = T(\vv) \)
    \item Calculate \(
        \id_W \left( T (\id_V(\vv)) \right) = T(\vv)\)
        where \(\id_W\) is the identity map on \(W\).
\end{enumerate}

\paragraph{TODO: commutative diagram}

\paragraph{Rank and Nullity of Matrices}
Let \(T: V \to W\) be a linear map between finite dimensional vector spaces
over \(\mathbb{F}\).Let \(A\) be the matrix for this transformation
with respect to any two bases in \(V\) and \(W\).

Then, \[
    \nullity(A) = \nullity(T) \text{ and } \rank(A) = \rank(T)
.\]

\paragraph{Invariant Subspaces}
Let \(V\) be a vector space over \( \mathbb{F} \) and \(T: V\to V\)
be a linear map.
If \(X \leq V\) such that \(T(X) \leq X\) then, \(X\) is an invariant
subspace of \(T\).

Recall that we know there is always a complementary subspace \(Y\)
such that \(V = X \oplus Y\).
Note, \(Y\) is not necessarily invariant.


