\section{Vector Spaces}

\paragraph{Motivation for Vector Spaces}
Vector spaces are a natural and important generalisation of 
\(\mathbb{R}^n\). It is natural to consider them whenever
it is possible to add objects and multiply them by scalars.

It may be convenient to consider a field \(\mathbb{F}\) as a vector
space over one of its subfields.

\paragraph{Definition of Vector Spaces}
Let \(\mathbb{F}\) be a field.
Then, a vector space over a field \(\mathbb{F}\) consists of an abelian
group \((V, +)\) and, a function from \(\mathbb{F}\times V \to V\) called
scalar multiplication and written as \(\alpha V\) where the following
properties hold.
\begin{enumerate}
    \item \textbf{Associativity over scalar multiplication:}
    \(\alpha (\beta v) = (\alpha\beta)v\)
    for all \(v\in V\), \(\alpha, \beta \in \mathbb{F}\)
    \item \textbf{Existence of 1:} \(1v = v\) for all \(v\in V\)
    \item \textbf{Distributivity of scalar multiplication over addition:}
    \(\alpha(u + v) = \alpha u + \alpha v\)
    for all \(u, v \in V, \alpha \in \mathbb{F}\)
    \item \textbf{Distributivity of addition over scalar multiplication:}
    \((\alpha+\beta)u = \alpha u + \beta u\)
\end{enumerate}



\paragraph{Properties and Notation for Vector Spaces}
\(\vec{dsdfa}\)
\begin{enumerate}
    \item Note that there are actually a total of ten axioms that exist.
    There is the four mentioned above, closure under scalar multiplication
    and, five that are inherited from the abelian group.
    \item Addition in \(V\) is called \textit{vector addition}
    to separate it from addition in \(\mathbb{F}\).
    \item \(V\) cannot be empty since it is a group.
    \item Bold face letters \(\vx, \vy, \vz\) may be used instead of \(x, y, z\),
    to denote vectors. More specifically, the identify of \((V, +)\)
    is denoted at \(\zero\) rather than the \( 0 \)
    that denotes a scalar in \(\mathbb{F}\).
    \item All the results from chapter 1 such as uniqueness of zero,
    negatives cancellation, \dots all apply for vector addition.
\end{enumerate}

\paragraph{Results on Combining Vectors Addition and Scalar}
Let \(V\) be a vector space over a field \(\mathbb{F}\).
Then for all \(\vu, \vv,\vw \in V\) and \(\lambda \in \mathbb{F}\),
\begin{enumerate}
    \item \(0 v = \zero \) and \(\lambda \zero = \zero\),
    \item \((-1)\vv = -\vv\),
    \item \(\lambda\vv = \zero \) implies either,
    \(\lambda = 0\) or \(\vv = 0\)
    \item If \(\lambda\vv = \lambda\vw\) where \(\lambda\neq 0\),
    then, \(\vv = \vw\).
\end{enumerate}

\subsection{Standard Examples of Vector Spaces}

\paragraph{The space \(\mathbb{F}^n\) over \(\mathbb{F}\)}
The set \(\mathbb{F}^n\) contains all \(n\)-tuples of elements of
\(\mathbb{F}\).
That is,
\[
    \mathbb{F}^n =
    \left\{ 
        \begin{pmatrix}
            \alpha_1 \\ \vdots \\ \alpha_n
        \end{pmatrix}
    \right\}
    :
    \alpha_i \in \mathbb{F}
\]

Let \(\vx = (\alpha_i)_{1\leq i \leq n}\)
and \(\vy = (\beta_i)_{1\leq i \leq n}\) be elements of \(\mathbb{F}\).
Then, vector addition is defined as
\[
        \vx + \vy = (\alpha_i + \beta_i)_{1\leq i \leq n}.
\]

Likewise, scalar multiplication on \(\mathbb{F}^n\) is defined as
\[
    \lambda\vx = (\lambda \vx_i)_{1 \leq i \leq n}.
\]

\paragraph{Geometric Vectors}
Geometric vectors are ordered pairs of points in \(\mathbb{R}^n\)
joined by label arrows. That is, they have direction and length.
These may be added by placing head to tail where, scalar multiplication
refers to increasing the length of the vector by a scalar value.

These vectors however, do not form a vector space. To do so, we define two
geometric vectors to be equivalent if one is a translation of the other.
Then, the set of equivalence classes of geometric vectors is a vector space.
That is, we do not care about the position of the geometric vector,
only its magnitude and direction.

\paragraph{Matrices}
For positive integers \(p, q\) the set \(M_{p, q}(\mathbb{F})\)
is the set of \(p\times q\) matrices with elements from \(F\).
Then, \(M_{p, q}\) is a vector space over \(\mathbb{F}\)
where vector addition and multiplication by a scalar are defined
by adding each corresponding element or, multiplying each element
by a the scalar.

\paragraph{Polynomials}
The set of all polynomials with coefficients in \(\mathbb{F}\)
denoted by \(\mathcal{P}(\mathbb{F})\) is a vector space over
\(\mathbb{F}\) with
\begin{align*}
    (f+g)(x) &= f(x) + g(x) && \text{ for all } x\in \mathbb{F}, \\
    (\lambda f)(x) &= \lambda f(x) && \text{ for all } \lambda, x\in \mathbb{F}.
\end{align*}

We may denote \(\mathcal{P}(\mathbb{F})\) to be the set of all polynomials
with degree \(n\) or less. This is also a vector space over \(\mathbb{F}\).

\paragraph{Function Spaces}
Let \(X\) be a non-empty set and \(\mathbb{F}\) be a field.
Then,
\[
    \mathcal{F}[X] = \left\{ f : X\to \mathbb{F} \right\}
\]
where \(\mathcal{F}[X]\)is a vector space of \(F\) representing the
set of all functions.
We must define
\begin{enumerate}
    \item The zero to be the zero function \(x \to 0\) for all \(x\in X\)
    \item \((f + g)(x) = f(x) + g(x)\) for all \(x\in X\)
    \item \((\lambda f)(x) = \lambda(f(x))\) for all \(x\in X\)
\end{enumerate}

Note that here, we use \(\mathcal{F}\) to correspond with \(\mathbb{F}\).
If we were however using \(\mathbb{R}\) as a field then, we may instead
prefer to use \(\mathcal{R}\) for the set of all functions instead.
Similarly, we extend this for \(\mathcal{Q}\) too.

\subsection{Subspaces}

\paragraph{Defining Vector Subspaces}
If \(V\) is a vector space over \(\mathbb{F}\) and \(U\subseteq V\)
then, \(U\) is a subspace of \(V\), denoted as \(U \leq V\)
if, it is a vector space over \(\mathbb{F}\) with the same addition
and scalar multiplication as \(V\).

Observe that every vector space has \(\{0\}\) (the trivial subspace)
and itself as subspaces.

\paragraph{Subspace Lemma}
To check if \(U\) is a subspace of a vector space \(V\), is is sufficient
to just check for closure under addition and scalar multiplication.
These conditions may be combined such that \(U\) is a subspace of
\(V\) if and only if, for all \(\alpha \in \mathbb{F}\),
\(\vu, \vv \in U\), \(\alpha\vu + \vv \in \mathbb{F}\).

The other axioms may be inherited from \(V\) and it must be ensured that
\(\zero \in U\).


